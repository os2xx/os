%%%%%%%%%%%%%%%%%%%%%%%%%%%%%%%%%%%%%%%%%%%%%%%%%%%%%%%%%%%%%%%%%%%%%%%%
% Beamer Presentation - LaTeX - Template Version 1.0 (10/11/12)
% This template has been downloaded from: http://www.LaTeXTemplates.com
% License: % CC BY-NC-SA 3.0 (http://creativecommons.org/)
% Modified by Rahmat M. Samik-Ibrahim

% REV319 Mon 19 Jul 2021 23:57:40 WIB
% REV303 Sun 02 May 2021 16:50:46 WIB
% REV258 Tue 24 Nov 2020 09:25:10 WIB
% REV239 Sun Sep 13 19:17:17 WIB 2020
% REV186 Wed Jan 30 14:55:14 WIB 2019
% STARTX Wed Sep 14 10:54:18 WIB 2016
%%%%%%%%%%%%%%%%%%%%%%%%%%%%%%%%%%%%%%%%%%%%%%%%%%%%%%%%%%%%%%%%%%%%%%%%%

% PACKAGES AND THEMES ZCZC
\documentclass[xcolor=table, notheorems, hyperref={pdfpagelabels=false}]{beamer}
\input{beamer.tex}
\newcommand{\revision}{REV362 21-Nov-2021}
% w! tmptmp
% REV362 Sun 21 Nov 2021 17:54:07 WIB
% REV359 Sat 30 Oct 2021 14:42:29 WIB
% REV349 Sun 26 Sep 2021 09:13:27 WIB
% REV339 Sat 04 Sep 2021 12:50:36 WIB
% REV329 Tue 17 Aug 2021 20:15:00 WIB
% STARTS Wed 24 Aug 2016 19:34:33 WIB
%%%%%%%%%%%%%%%%%%%%%%%%%%%%%%%%%%%%%
\newcommand{\kopikopi}{\textcopyright{}2016-2021 VauLSMorg}



% XXXXXXXXXXXXXXXXXXXXXXXXXXXXXXXXXXXXXXXXXXXXXXXXXXXXXXXXXXXXXXXXXXXXXXXXXX
% The short title appears at the bottom of every slide, 
% the full title is only on the title page
% \date{\today}
\title[\kopikopi]
{CSGE602055 Operating Systems \\ 
CSF2600505 Sistem Operasi \\
Week 08:
Scheduling}
\author{Rahmat M. Samik-Ibrahim (ed.)}
\institute[UI]
{
University of Indonesia \\
\medskip
\url{https://os.vlsm.org/Slides/os08.pdf}
\\ \texttt{Always check for the latest revision!}
}
\date{\revision}

% XXXXXXXXXXXXXXXXXXXXXXXXXXXXXXXXXXXXXXXXXXXXXXXXXXXXXXXXXXXXXXXXXXXXXXXXXX
\begin{document}

\lstset{
basicstyle=\ttfamily\tiny, % \tiny \small \footnotesize 
breakatwhitespace=true,
language=C,
columns=fullflexible,
keepspaces=true,
breaklines=true,
tabsize=3, 
showstringspaces=false,
extendedchars=true}
\section{Start}
\begin{frame}
\titlepage
\end{frame}

% XXXXXXXXXXXXXXXXXXXXXXXXXXXXXXXXXXXXXXXXXXXXXXXXXXXXXXXXXXXXXXXXXXXXXXXXXX

%%%%%%%%%%%%%%%%%%%%%%%%%%%%%%%%%%%%%%%%%%%%%%%%%%%%%%%%%%%%%%%%%%%%%%%%%
% REV352 Sun 10 Oct 2021 09:56:47 WIB
% REV341 Sun 05 Sep 2021 23:30:00 WIB
% REV333 Thu 26 Aug 2021 08:52:24 WIB
% REV328 Sat 14 Aug 2021 06:32:08 WIB
% REV272 Mon 01 Mar 2021 12:02:09 WIB
% START0 Sat Sep  2 10:51:33 WIB 2017
%%%%%%%%%%%%%%%%%%%%%%%%%%%%%%%%%%%%%%%%%%%%%%%%%%%%%%%%%%%%%%%%%%%%%%%%%

\begin{frame}[fragile]
\section{Schedule}
\frametitle{OS212\footnote{%
) This information will be on \textbf{EVERY} page two (2) of this course material.}): 
Operating Systems 2021 - 2}
\scalebox{0.73}{%
\begin{tabular}{|c|c|c|c|}
\hline
\makebox[106pt]{OS A} & \makebox[106pt]{OS B} & \makebox[107pt]{OS C} & \makebox[107pt]{OS INT} \\
\hline
\multicolumn{4}{|c|}{Every first day of the Week, \textbf{Quiz\#1:} (07:40-07:50) and \textbf{Quiz\#2:} 07:20-07:40} \\
\hline
Monday/Thursday & Monday/Thursday & Monday/Thursday & Monday/Wednesday   \\
13:00 --- 14:40  & 15:00 --- 16:40\footnote{) \textbf{OS B:} Week00-Week05 (RMS); Week06-Week10 (MAM).} &
                                      13:00 --- 14:40 & 08:00 --- 09:40  \\
14:00 --- finish & 16:00 --- finish & 13:00 --- 14:40 & 09:00 --- finish \\
\hline
\end{tabular}
}

\vspace{5pt}

\scalebox{0.73}{%
\begin{tabular}{|c|c|l|l|}
\hline
\textbf{Week} & \textbf{Schedule \& Deadline}\footnote{%
    ) The \textbf{DEADLINE} of Week 00 is 05 Sep 2021,
      whereas the \textbf{DEADLINE} of Week 01 is 12 Sep 2021, and so on...%
    })& \textbf{Topic} & \textbf{OSC10}\footnote{%
    ) Silberschatz et. al.: \textbf{Operating System Concepts}, $10^{th}$ Edition, 2018.}) \\
\hline
Week 00  & 30 Aug - 05 Sep 2021 & Overview 1, Virtualization \& Scripting & Ch. 1, 2, 18. \\
Week 01  & 06 Sep - 12 Sep 2021 & Overview 2, Virtualization \& Scripting & Ch. 1, 2, 18. \\
Week 02  & 13 Sep - 19 Sep 2021 & Security, Protection, Privacy, \& C-language.  & Ch. 16, 17. \\
Week 03  & 20 Sep - 26 Sep 2021 & File System \& FUSE  & Ch. 13, 14, 15. \\
Week 04  & 27 Sep - 03 Oct 2021 & Addressing, Shared Lib, \& Pointer & Ch. 9. \\
Week 05  & 04 Oct - 10 Oct 2021 & Virtual Memory & Ch. 10. \\
\hline
Week 06  & 11 Oct - 31 Oct 2021 & Concurrency: Processes \& Threads & Ch. 3, 4. \\
Week 07  & 01 Nov - 07 Nov 2021 & Synchronization \& Deadlock & Ch. 6, 7, 8. \\
Week 08  & 08 Nov - 14 Nov 2021 & Scheduling + W06/W07 & Ch. 5. \\
Week 09  & 15 Nov - 21 Nov 2021 & Storage, Firmware, Bootloader, \& Systemd & Ch. 11. \\
Week 10  & 22 Nov - 28 Nov 2021 & I/O \& Programming & Ch. 12. \\%
% MidTerm  & 00 XXX 2020 (XX:XX-XX:XX) & MidTerm (UTS) & \cellcolor{red!44} TBA! \\
% Reserved & 00 XXX - 00 XXX 2020 & Q \& A & \\
% Final    & 00 XXX 2020 XX:XX & First Part Final  (UAS tahap I)  & \cellcolor{red!44} This schedule is   \\
% Extra    & NA & No Extra assignment & \cellcolor{red!44} subject to change. \\
\hline
\end{tabular}
}
\end{frame}

\begin{frame}[fragile]
\frametitle{\textbf{STARTING POINT} --- 
{
\definecolor{links}{HTML}{FDEE00}
\hypersetup{colorlinks,linkcolor=,urlcolor=links}
\url{https://os.vlsm.org/}
}
}
\begin{itemize}
\item[$\square$] \textbf{Text Book} ---
                 Any recent/decent OS book. Eg. (\textbf{OSC10}) Silberschatz et. al.: 
                 \textbf{Operating System Concepts}, $10^{th}$ Edition, 2018.
                 See also \url{https://www.os-book.com/OS10/}.
\item[$\square$] \textbf{Resources}
\begin{itemize}
\item[$\square$] \href{https://scele.cs.ui.ac.id/course/view.php?id=3268}{\textbf{SCELE OS212}} ---
\url{https://scele.cs.ui.ac.id/course/view.php?id=3268}.\\
The enrollment key is \textbf{XXX}.
\item[$\square$] \textbf{Download Slides and Demos from GitHub.com} \\
\url{https://github.com/UI-FASILKOM-OS/SistemOperasi/}:

                 {\scriptsize%
                 \href{https://os.vlsm.org/Slides/os00.pdf}{\texttt{os00.pdf} (W00)},
                 \href{https://os.vlsm.org/Slides/os01.pdf}{\texttt{os01.pdf} (W01)},
                 \href{https://os.vlsm.org/Slides/os02.pdf}{\texttt{os02.pdf} (W02)},
                 \href{https://os.vlsm.org/Slides/os03.pdf}{\texttt{os03.pdf} (W03)},

                 \href{https://os.vlsm.org/Slides/os04.pdf}{\texttt{os04.pdf} (W04)},
                 \href{https://os.vlsm.org/Slides/os05.pdf}{\texttt{os05.pdf} (W05)},
                 \href{https://os.vlsm.org/Slides/os06.pdf}{\texttt{os06.pdf} (W06)},
                 \href{https://os.vlsm.org/Slides/os07.pdf}{\texttt{os07.pdf} (W07)},

                 \href{https://os.vlsm.org/Slides/os08.pdf}{\texttt{os08.pdf} (W08)},
                 \href{https://os.vlsm.org/Slides/os09.pdf}{\texttt{os09.pdf} (W09)},
                 \href{https://os.vlsm.org/Slides/os10.pdf}{\texttt{os10.pdf} (W10)}.
                 }
\item[$\square$] \textbf{Problems}\\
                 {\scriptsize% 
                 \href{https://rms46.vlsm.org/2/195.pdf}{\texttt{195.pdf} (W00)},
                 \href{https://rms46.vlsm.org/2/196.pdf}{\texttt{196.pdf} (W01)},
                 \href{https://rms46.vlsm.org/2/197.pdf}{\texttt{197.pdf} (W02)},
                 \href{https://rms46.vlsm.org/2/198.pdf}{\texttt{198.pdf} (W03)},\\
                 \href{https://rms46.vlsm.org/2/199.pdf}{\texttt{199.pdf} (W04)},
                 \href{https://rms46.vlsm.org/2/200.pdf}{\texttt{200.pdf} (W05)},
                 \href{https://rms46.vlsm.org/2/201.pdf}{\texttt{201.pdf} (W06)},
                 \href{https://rms46.vlsm.org/2/202.pdf}{\texttt{202.pdf} (W07)},\\
                 \href{https://rms46.vlsm.org/2/203.pdf}{\texttt{203.pdf} (W08)},
                 \href{https://rms46.vlsm.org/2/204.pdf}{\texttt{204.pdf} (W09)},
                 \href{https://rms46.vlsm.org/2/205.pdf}{\texttt{205.pdf} (W10)}.}
\item[$\square$] \textbf{LFS} --- \url{http://www.linuxfromscratch.org/lfs/view/stable/}
\item[$\square$] \textbf{OSP4DISS} --- \url{https://osp4diss.vlsm.org/}
\item[$\square$] \textbf{DOIT} --- \url{https://doit.vlsm.org/001.html}
\end{itemize}
\end{itemize}
\end{frame}



% XXXXXXXXXXXXXXXXXXXXXXXXXXXXXXXXXXXXXXXXXXXXXXXXXXXXXXXXXXXXXXXXXXXXXXXXXX
\section{Agenda}
\begin{frame}[fragile]
\frametitle{Agenda}
% Throughout your presentation, if you choose to use \section{} and 
% \subsection{} commands, these will automatically be printed on 
% this slide as an overview of your presentation
\tableofcontents 
\end{frame}

% XXXXXXXXXXXXXXXXXXXXXXXXXXXXXXXXXXXXXXXXXXXXXXXXXXXXXXXXXXXXXXXXXXXXXXXXXX

%%%%%%%%%%%%%%%%%%%%%%%%%%%%%%%
% REV412: Tue 22 Aug 2023 14:00
% REV406: Sat 05 Aug 2023 12:00
% REV154: Thu 23 Aug 2018 11:00
% START0: Thu 26 Jul 2018 20:00
%%%%%%%%%%%%%%%%%%%%%%%%%%%%%%%

\section{Week 08}
\begin{frame}[fragile]
\frametitle{Week 08 Scheduling:
Topics\footnote{Source: ACM IEEE CS Curricula}}

\begin{itemize}
\item Preemptive and non-preemptive scheduling 
\item Schedulers and policies
\item Processes and threads
\item Deadlines and real-time issues
\end{itemize}
\end{frame}

\begin{frame}[fragile]
\frametitle{Week 08 Scheduling:
Learning Outcomes\footnote{Source: ACM IEEE CS Curricula}}
\begin{itemize}
\item Compare and contrast the common algorithms used for both preemptive and non-preemptive scheduling of tasks in operating systems, such as priority, performance comparison, and fair-share schemes. [Usage]
\item Describe relationships between scheduling algorithms and application domains. [Familiarity]
\item Discuss the types of processor scheduling such as short-term, medium-term, long-term, and I/O.  [Familiarity]
\item Describe the difference between processes and threads. [Usage]
\item Compare and contrast static and dynamic approaches to real-time scheduling. [Usage]
\item Discuss the need for preemption and deadline scheduling. [Familiarity]
\item Identify ways that the logic embodied in scheduling algorithms are applicable to other domains, such as disk I/O, network scheduling, project scheduling, and problems beyond computing. [Usage]
\end{itemize}

\end{frame}



% XXXXXXXXXXXXXXXXXXXXXXXXXXXXXXXXXXXXXXXXXXXXXXXXXXXXXXXXXXXXXXXXXXXXXXXXXX
\section{Scheduling}
\begin{frame}
\frametitle{Week 08: Scheduling}
\begin{itemize}
\item Reference: (OSC10-ch05 demo-w08)
\item Scheduling
\begin{itemize}
\item Basic Concepts
\begin{itemize}
\item \textbf{WARNING:} It's just a BURST
\item IO Burst
\item CPU Burst
\item CPU Burst vs. Freq (See next slide)
\end{itemize}
\item Criteria: Utilization, throughput, \{turnaround, waiting,  response\} time.
\item (Burst) Algorithm
\begin{itemize}
\item FCFS, SJF, RR, Priority, Multilevel Queue.
\end{itemize}
\item Preemptive / Non-preemptive (Cooperative) Scheduling
\item I/O Bound / CPU Bound Processes
\end{itemize}
\item Thread Scheduling
\begin{itemize}
\item User-level $\rightarrow$ Process-Contention Scope (PCS): many to many/one.
\item Kernel-level $\rightarrow$ System-Contention Scope (SCS): one to one.
\end{itemize}
\item Standard Linux Scheduling
\begin{itemize}
\item Completely Fair Scheduler (CFS).
\item Real Time Scheduling.
\end{itemize}
\end{itemize}
\end{frame}

% XXXXXXXXXXXXXXXXXXXXXXXXXXXXXXXXXXXXXXXXXXXXXXXXXXXXXXXXXXXXXXXXXXXXXXXXXX
\section{CPU Burst: How Long (When)?}
\begin{frame}
\frametitle{CPU Burst: How Long (When)?}
\begin{figure}
\includegraphics[width=0.80\linewidth]{os08-osc9}
\caption{Burst: Duration vs Frequency}
\end{figure}
\end{frame}

% XXXXXXXXXXXXXXXXXXXXXXXXXXXXXXXXXXXXXXXXXXXXXXXXXXXXXXXXXXXXXXXXXXXXXXXXXX
\section{MultiProcessor Schedulling}
\begin{frame}
\frametitle{MultiProcessor Schedulling}
\begin{itemize}
\item Asymmetric Multiprocessing vs. Symmetric Multiprocessing (SMP).
\item Processor Affinity: soft vs. hard.
\item NUMA: Non-Uniform Memory Access.
\item Load Balancing
\item Multicore Processors
\item Real Time Schedulling: Soft vs. Hard.
\item Big O Notation
\begin{itemize}
\item O(1)
\item O(log N)
\item O(N)
\end{itemize}
\end{itemize}
\end{frame}

% XXXXXXXXXXXXXXXXXXXXXXXXXXXXXXXXXXXXXXXXXXXXXXXXXXXXXXXXXXXXXXXXXXXXXXXXXX
\section{The Two State Model}
\begin{frame}
\frametitle{The Two State Model}
\begin{itemize}
\item CPU State -- I/O State -- CPU State -- \dots
\begin{itemize}
\item n: processes in memory.
\item p: I/O time fraction.
\item $ p^n $: probability n processes waiting for I/O.
\item $ 1 - p^n $: CPU utilization of n processes.
\item $ \left[\frac{\left(1 - p^n\right)}{n}\right] $: CPU utilization of ONE processes.
\end{itemize}
\item Example: $ p = 60 \% \Rightarrow $ \textbf{CPU Utilization Per Process}: $ \left[\frac{1 - \left(60\%\right)^n}{n}\right] $
\\[10pt]
\begin{tabular}{ | c | c | c | c | c | c | }
\hline
\textbf{CPU Utilization}
&
\multicolumn{5}{| c |}{\textbf{Multiprogramming (\%)}} \\
\hline
\textbf{N}
&
\makebox[17pt][c]{\textbf{1}}
&
\makebox[17pt][c]{\textbf{2}}
&
\makebox[17pt][c]{\textbf{3}}
&
\makebox[17pt][c]{\textbf{4}}
&
\makebox[17pt][c]{\textbf{5}}
\\
\hline
\textbf{Per Process} & 40 & 32 & 26 & 21 & 18 \\
\hline
\end{tabular}
\\[10pt]
\begin{itemize}
\item For 5 concurrent processes: \\
If total time is 100 seconds; for each processs, the CPU time will be 18 seconds.
\end{itemize}
\end{itemize}
\end{frame}

% XXXXXXXXXXXXXXXXXXXXXXXXXXXXXXXXXXXXXXXXXXXXXXXXXXXXXXXXXXXXXXXXXXXXXXXXXX

% %%%%%%%%%%%%%%%%%%%%%%%%%%%%%%%%%%%%%%%%%%%%%%%%%%%%%%%%%%%%%%%%%%%%%%%
% Beamer Presentation - LaTeX - Template Version 1.0 (10/11/12)
% This template has been downloaded from: http://www.LaTeXTemplates.com
% License: % CC BY-NC-SA 3.0 (http://creativecommons.org/)
% Modified by Rahmat M. Samik-Ibrahim

% REV360 Sun 07 Nov 2021 21:35:08 WIB
% REV309 Mon 05 Jul 2021 15:32:10 WIB
% REV304 Mon 03 May 2021 01:00:47 WIB
% REV303 Sun 02 May 2021 16:50:46 WIB
% REV300 Sun 18 Apr 2021 05:50:56 WIB
% STARTX Sun 13 Sep 2020 08:49:47 WIB
% %%%%%%%%%%%%%%%%%%%%%%%%%%%%%%%%%%%%%%%%%%%%%%%%%%%%%%%%%%%%%%%%%%%%%%%%

% XXXXXXXXXXXXXXXXXXXXXXXXXXXXXXXXXXXXXXXXXXXXXXXXXXXXXXXXXXXXXXXXXXXXXXXXXX
\section{Week 08: Check List}
\begin{frame}
\frametitle{Week 08: Check List (Deadline: 14 Nov 2021).}
\begin{itemize}
\item [$\square$] Week 08: Assignment (\href{https://os.vlsm.org/Slides/os08.pdf}{\textbf{os08.pdf}}).
(Eg. \textbf{cbkadal}).
\begin{itemize}
\item Visit \url{https://osp4diss.vlsm.org/\#idx0708}
\item Week 08 - 10 will be about building ''Linux From Scratch (LFS)''
\begin{enumerate}
\item \href{https://www.os-book.com/OS10/slide-dir/}{Read OSC10 chapter 5}
\item Try Demos in {\tiny \url{https://github.com/UI-FASILKOM-OS/SistemOperasi/tree/master/Demos/}}.
\item Try Previous FinalTerm Problems {\tiny (\url{https://rms46.vlsm.org/2/203.pdf})}.
\item \href{https://osp4diss.vlsm.org/W08-01.html}{Linux From Scratch 11.0 Chapter 01-04} \\
(a) \href{https://os.vlsm.org/WEEK/WEEK08.tar.bz2.asc}{Fetch and Extract File WEEK08.tar.bz2.asc}.\\
(b) Follow the official LFS guide version 11.0.
\item \href{https://osp4diss.vlsm.org/W08-02.html}{Linux From Scratch 11.0 Chapter 05} \\
(a) Chapter 05 ''Compiling a Cross-Toolchain''
\item \href{https://cbkadal.github.io/os212/LINKS/}{Update your bookmark links. See C.B. Kadal's "LINKS/"}.
\item \href{https://cbkadal.github.io/os212/TIPS/}{(Optional) Any suggestions/tips for the next semester class? See C.B. Kadal's "TIPS/"}.
\item \href{https://osp4diss.vlsm.org/W02-05.html}{Review your peer links}.
\item \href{https://cbkadal.github.io/os212/TXT/mylog.txt}{Update your log. See C.B. Kadal's ''mylog.txt''}
\item Submit your Week 08 Assignment (\href{https://osp4diss.vlsm.org/W03-06.html}{See Week 03}).
\end{enumerate}
\end{itemize}
\end{itemize}
\end{frame}



% 12 XXXXXXXXXXXXXXXXXXXXXXXXXXXXXXXXXXXXXXXXXXXXXXXXXXXXXXXXXXXXXXXXXXXXXXX
% XXXXXXXXXXXXXXXXXXXXXXXXXXXXXXXXXXXXXXXXXXXXXXXXXXXXXXXXXXXXXXXXXXXXXXXXXX
\section{The End}
\begin{frame}
\frametitle{The End}
\begin{itemize}
\item[$\square$] This is the end of the presentation.
\item[$\boxtimes$] This is the end of the presentation.
\item This is the end of the presentation.
\end{itemize}
\end{frame}

% XXXXXXXXXXXXXXXXXXXXXXXXXXXXXXXXXXXXXXXXXXXXXXXXXXXXXXXXXXXXXXXXXXXXXXXXXX
\end{document}

