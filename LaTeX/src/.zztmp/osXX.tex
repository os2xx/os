%%%%%%%%%%%%%%%%%%%%%%%%%%%%%%%%%%%%%%%%%%%%%%%%%%%%%%%%%%%%%%%%%%%%%%%%
% Beamer Presentation - LaTeX - Template Version 1.0 (10/11/12)
% This template has been downloaded from: http://www.LaTeXTemplates.com
% License: % CC BY-NC-SA 3.0 (http://creativecommons.org/)
% Modified by Rahmat M. Samik-Ibrahim
% REV213 Thu Jan 23 14:02:18 WIB 2020
% REV189 Thu Feb  7 09:17:02 WIB 2019
% REV116 Wed Feb  7 19:00:08 WIB 2018
% REV044 Thu Apr 27 11:22:16 WIB 2017
% STARTX Wed Aug 24 19:34:33 WIB 2016
%%%%%%%%%%%%%%%%%%%%%%%%%%%%%%%%%%%%%%%%%%%%%%%%%%%%%%%%%%%%%%%%%%%%%%%%%

% PACKAGES AND THEMES ZCZC
\documentclass[xcolor=table, notheorems, hyperref={pdfpagelabels=false}]{beamer}
\input{beamer.tex}
\newcommand{\revision}{REV362 21-Nov-2021}
% w! tmptmp
% REV362 Sun 21 Nov 2021 17:54:07 WIB
% REV359 Sat 30 Oct 2021 14:42:29 WIB
% REV349 Sun 26 Sep 2021 09:13:27 WIB
% REV339 Sat 04 Sep 2021 12:50:36 WIB
% REV329 Tue 17 Aug 2021 20:15:00 WIB
% STARTS Wed 24 Aug 2016 19:34:33 WIB
%%%%%%%%%%%%%%%%%%%%%%%%%%%%%%%%%%%%%
\newcommand{\kopikopi}{\textcopyright{}2016-2021 VauLSMorg}



% XXXXXXXXXXXXXXXXXXXXXXXXXXXXXXXXXXXXXXXXXXXXXXXXXXXXXXXXXXXXXXXXXXXXXXXXXX
% The short title appears at the bottom of every slide, 
% the full title is only on the title page
% \date{\today}
\title[\kopikopi]
{CSF2600505 Sistem Operasi \\
CSGE602055 Operating Systems \\ 
osXX: The Lecturer} 
\author{Rahmat M. Samik-Ibrahim (ed.)}
\institute[UI]
{
University of Indonesia\\ 
\medskip
\url{https://os.vlsm.org/}
\\ \texttt{Always check for the latest revision!}
}
\date{\revision}

% XXXXXXXXXXXXXXXXXXXXXXXXXXXXXXXXXXXXXXXXXXXXXXXXXXXXXXXXXXXXXXXXXXXXXXXXXX
\begin{document}

\lstset{
basicstyle=\ttfamily\tiny, % \tiny \small \footnotesize 
breakatwhitespace=true,
language=C,
columns=fullflexible,
keepspaces=true,
breaklines=true,
tabsize=3, 
showstringspaces=false,
extendedchars=true}

\section{Start}
\begin{frame}
\titlepage
\end{frame}

% XXXXXXXXXXXXXXXXXXXXXXXXXXXXXXXXXXXXXXXXXXXXXXXXXXXXXXXXXXXXXXXXXXXXXXXXXX

%%%%%%%%%%%%%%%%%%%%%%%%%%%%%%%%%%%%%%%%%%%%%%%%%%%%%%%%%%%%%%%%%%%%%%%%%
% REV352 Sun 10 Oct 2021 09:56:47 WIB
% REV341 Sun 05 Sep 2021 23:30:00 WIB
% REV333 Thu 26 Aug 2021 08:52:24 WIB
% REV328 Sat 14 Aug 2021 06:32:08 WIB
% REV272 Mon 01 Mar 2021 12:02:09 WIB
% START0 Sat Sep  2 10:51:33 WIB 2017
%%%%%%%%%%%%%%%%%%%%%%%%%%%%%%%%%%%%%%%%%%%%%%%%%%%%%%%%%%%%%%%%%%%%%%%%%

\begin{frame}[fragile]
\section{Schedule}
\frametitle{OS212\footnote{%
) This information will be on \textbf{EVERY} page two (2) of this course material.}): 
Operating Systems 2021 - 2}
\scalebox{0.73}{%
\begin{tabular}{|c|c|c|c|}
\hline
\makebox[106pt]{OS A} & \makebox[106pt]{OS B} & \makebox[107pt]{OS C} & \makebox[107pt]{OS INT} \\
\hline
\multicolumn{4}{|c|}{Every first day of the Week, \textbf{Quiz\#1:} (07:40-07:50) and \textbf{Quiz\#2:} 07:20-07:40} \\
\hline
Monday/Thursday & Monday/Thursday & Monday/Thursday & Monday/Wednesday   \\
13:00 --- 14:40  & 15:00 --- 16:40\footnote{) \textbf{OS B:} Week00-Week05 (RMS); Week06-Week10 (MAM).} &
                                      13:00 --- 14:40 & 08:00 --- 09:40  \\
14:00 --- finish & 16:00 --- finish & 13:00 --- 14:40 & 09:00 --- finish \\
\hline
\end{tabular}
}

\vspace{5pt}

\scalebox{0.73}{%
\begin{tabular}{|c|c|l|l|}
\hline
\textbf{Week} & \textbf{Schedule \& Deadline}\footnote{%
    ) The \textbf{DEADLINE} of Week 00 is 05 Sep 2021,
      whereas the \textbf{DEADLINE} of Week 01 is 12 Sep 2021, and so on...%
    })& \textbf{Topic} & \textbf{OSC10}\footnote{%
    ) Silberschatz et. al.: \textbf{Operating System Concepts}, $10^{th}$ Edition, 2018.}) \\
\hline
Week 00  & 30 Aug - 05 Sep 2021 & Overview 1, Virtualization \& Scripting & Ch. 1, 2, 18. \\
Week 01  & 06 Sep - 12 Sep 2021 & Overview 2, Virtualization \& Scripting & Ch. 1, 2, 18. \\
Week 02  & 13 Sep - 19 Sep 2021 & Security, Protection, Privacy, \& C-language.  & Ch. 16, 17. \\
Week 03  & 20 Sep - 26 Sep 2021 & File System \& FUSE  & Ch. 13, 14, 15. \\
Week 04  & 27 Sep - 03 Oct 2021 & Addressing, Shared Lib, \& Pointer & Ch. 9. \\
Week 05  & 04 Oct - 10 Oct 2021 & Virtual Memory & Ch. 10. \\
\hline
Week 06  & 11 Oct - 31 Oct 2021 & Concurrency: Processes \& Threads & Ch. 3, 4. \\
Week 07  & 01 Nov - 07 Nov 2021 & Synchronization \& Deadlock & Ch. 6, 7, 8. \\
Week 08  & 08 Nov - 14 Nov 2021 & Scheduling + W06/W07 & Ch. 5. \\
Week 09  & 15 Nov - 21 Nov 2021 & Storage, Firmware, Bootloader, \& Systemd & Ch. 11. \\
Week 10  & 22 Nov - 28 Nov 2021 & I/O \& Programming & Ch. 12. \\%
% MidTerm  & 00 XXX 2020 (XX:XX-XX:XX) & MidTerm (UTS) & \cellcolor{red!44} TBA! \\
% Reserved & 00 XXX - 00 XXX 2020 & Q \& A & \\
% Final    & 00 XXX 2020 XX:XX & First Part Final  (UAS tahap I)  & \cellcolor{red!44} This schedule is   \\
% Extra    & NA & No Extra assignment & \cellcolor{red!44} subject to change. \\
\hline
\end{tabular}
}
\end{frame}

\begin{frame}[fragile]
\frametitle{\textbf{STARTING POINT} --- 
{
\definecolor{links}{HTML}{FDEE00}
\hypersetup{colorlinks,linkcolor=,urlcolor=links}
\url{https://os.vlsm.org/}
}
}
\begin{itemize}
\item[$\square$] \textbf{Text Book} ---
                 Any recent/decent OS book. Eg. (\textbf{OSC10}) Silberschatz et. al.: 
                 \textbf{Operating System Concepts}, $10^{th}$ Edition, 2018.
                 See also \url{https://www.os-book.com/OS10/}.
\item[$\square$] \textbf{Resources}
\begin{itemize}
\item[$\square$] \href{https://scele.cs.ui.ac.id/course/view.php?id=3268}{\textbf{SCELE OS212}} ---
\url{https://scele.cs.ui.ac.id/course/view.php?id=3268}.\\
The enrollment key is \textbf{XXX}.
\item[$\square$] \textbf{Download Slides and Demos from GitHub.com} \\
\url{https://github.com/UI-FASILKOM-OS/SistemOperasi/}:

                 {\scriptsize%
                 \href{https://os.vlsm.org/Slides/os00.pdf}{\texttt{os00.pdf} (W00)},
                 \href{https://os.vlsm.org/Slides/os01.pdf}{\texttt{os01.pdf} (W01)},
                 \href{https://os.vlsm.org/Slides/os02.pdf}{\texttt{os02.pdf} (W02)},
                 \href{https://os.vlsm.org/Slides/os03.pdf}{\texttt{os03.pdf} (W03)},

                 \href{https://os.vlsm.org/Slides/os04.pdf}{\texttt{os04.pdf} (W04)},
                 \href{https://os.vlsm.org/Slides/os05.pdf}{\texttt{os05.pdf} (W05)},
                 \href{https://os.vlsm.org/Slides/os06.pdf}{\texttt{os06.pdf} (W06)},
                 \href{https://os.vlsm.org/Slides/os07.pdf}{\texttt{os07.pdf} (W07)},

                 \href{https://os.vlsm.org/Slides/os08.pdf}{\texttt{os08.pdf} (W08)},
                 \href{https://os.vlsm.org/Slides/os09.pdf}{\texttt{os09.pdf} (W09)},
                 \href{https://os.vlsm.org/Slides/os10.pdf}{\texttt{os10.pdf} (W10)}.
                 }
\item[$\square$] \textbf{Problems}\\
                 {\scriptsize% 
                 \href{https://rms46.vlsm.org/2/195.pdf}{\texttt{195.pdf} (W00)},
                 \href{https://rms46.vlsm.org/2/196.pdf}{\texttt{196.pdf} (W01)},
                 \href{https://rms46.vlsm.org/2/197.pdf}{\texttt{197.pdf} (W02)},
                 \href{https://rms46.vlsm.org/2/198.pdf}{\texttt{198.pdf} (W03)},\\
                 \href{https://rms46.vlsm.org/2/199.pdf}{\texttt{199.pdf} (W04)},
                 \href{https://rms46.vlsm.org/2/200.pdf}{\texttt{200.pdf} (W05)},
                 \href{https://rms46.vlsm.org/2/201.pdf}{\texttt{201.pdf} (W06)},
                 \href{https://rms46.vlsm.org/2/202.pdf}{\texttt{202.pdf} (W07)},\\
                 \href{https://rms46.vlsm.org/2/203.pdf}{\texttt{203.pdf} (W08)},
                 \href{https://rms46.vlsm.org/2/204.pdf}{\texttt{204.pdf} (W09)},
                 \href{https://rms46.vlsm.org/2/205.pdf}{\texttt{205.pdf} (W10)}.}
\item[$\square$] \textbf{LFS} --- \url{http://www.linuxfromscratch.org/lfs/view/stable/}
\item[$\square$] \textbf{OSP4DISS} --- \url{https://osp4diss.vlsm.org/}
\item[$\square$] \textbf{DOIT} --- \url{https://doit.vlsm.org/001.html}
\end{itemize}
\end{itemize}
\end{frame}



% XXXXXXXXXXXXXXXXXXXXXXXXXXXXXXXXXXXXXXXXXXXXXXXXXXXXXXXXXXXXXXXXXXXXXXXXXX
\section{Agenda}
\begin{frame}
\frametitle{Agenda}
% Throughout your presentation, if you choose to use \section{} and 
% \subsection{} commands, these will automatically be printed on 
% this slide as an overview of your presentation
\tableofcontents 
\end{frame}

% XXXXXXXXXXXXXXXXXXXXXXXXXXXXXXXXXXXXXXXXXXXXXXXXXXXXXXXXXXXXXXXXXXXXXXXXXX
\section{About and how to contact the Lecturer}
\begin{frame}
\frametitle{About and how to contact the Lecturer}
\begin{itemize}
\item Rahmat M. Samik-Ibrahim
\item Bekerja di Universitas Indonesia: sejak 1984\footnote{%
      MDCCXXXIV --- Universitas Goettingen didirikan: \textbf{1734}}.
\item Pengguna GNU/Linux: sejak 1994.
\item VauLSMorg (vlsm.org): sejak 1996.
\item Blog: \texttt{\url{https://rahmatm.samik-ibrahim.vlsm.org/}}
\begin{itemize}
\item Blog: \texttt{\textbf{2016/08/panggil-saya-rahmat.html}}
\item Blog: \texttt{\textbf{2013/10/kumpulan-hal.html}}
\item Blog: \texttt{\textbf{2011/08/ibu-ke-pasar-membeli-ayam.html}}
\end{itemize}
\item (Almost) no social media!
\begin{itemize}
\item Page Only: No need to be ''\textbf{ADDED}''
\item You can ''\textbf{Like}'' it, ''\textbf{Follow}'' it, or ''\textbf{Share}'' it.
\end{itemize}
\item \begin{tabular}{|c|}
\hline
For Q \& A, use WhatsApp Group \textbf{OperatingSystems}\\
(info +62-881-456-\textbf{XXXX}) \\
Email (Subject:\textbf{[HELP]}) operatingsystems@vlsm.org\\
\hline
\end{tabular}
\end{itemize}
\end{frame}

% XXXXXXXXXXXXXXXXXXXXXXXXXXXXXXXXXXXXXXXXXXXXXXXXXXXXXXXXXXXXXXXXXXXXXXXXXX
\begin{frame}
\frametitle{TOP 10 REALITA}
\begin{enumerate}
\item Nama saya \textbf{Rahmat}. Rahmat nama saya.
      Kalau bukan Rahmat, bukan nama saya!
\item Jangan datang lebih lambat dari pada Pengajar!
      Terdapat dua kesempatan untuk menyusul masuk kelas: T+15 menit dan setelah istirahat.
\item Jangan berisik/asyik sendiri dalam kelas dan jangan main 
      "\textit{games}" dan "\textit{chat}" dengan "\textit{gadget}" anda!
\item Jangan lupa mempersiapkan diri untuk berpartisipasi dalam kelas!
\item Jangan lupa membawa selembar kertas (+QRC) untuk membuat memo kuliah!
\item Memo kuliah (+QRC) tersebut yang boleh dibawa saat UTS dan UAS.
\item Jangan curang!
\item Jangan meminjam peralatan selama kuis dan ujian!
\item Jangan menghubungi Pengajar untuk masalah Administratip!
\item Jangan menjadi "\textit{Puss in Boot}"!
\end{enumerate}
\end{frame}

% XXXXXXXXXXXXXXXXXXXXXXXXXXXXXXXXXXXXXXXXXXXXXXXXXXXXXXXXXXXXXXXXXXXXXXXXXX
\begin{frame}
\frametitle{Jangan menjadi Puss In Boot}
\begin{figure}
\includegraphics[width=0.65\linewidth]{os00-pib}
\caption{Ini Puss in Boot\footnote{
   This is a fair use of a DreamWorks/Paramount Picture character.}.}
\end{figure}
\end{frame}

% XXXXXXXXXXXXXXXXXXXXXXXXXXXXXXXXXXXXXXXXXXXXXXXXXXXXXXXXXXXXXXXXXXXXXXXXXX
\section{The End (Ignore It)}
\begin{frame}[fragile]
\frametitle{The End (Ignore It)}

\begin{block}{GAUDEAMUS IGITUR}
Gaudeamus igitur. Iuvenes dum sumus. Post iucundam iuventutem. Post molestam senectutem. Nos habebit humus. 
Ubi sunt qui ante nos. In mundo fuere? Vadite ad superos. Transite in inferos. Hos si vis videre. 
\end{block}

\begin{lstlisting}[basicstyle=\ttfamily\small]

#include <stdio.h>

void main() {
   printf("Vita nostra brevis est. Brevi finietur.\n");
   printf("Venit mors velociter. Rapit nos atrociter.\n");
   printf("Nemini parcetur.\n\n");
}

\end{lstlisting}

\begin{itemize}
\item[$\square$]   Vivat academia! Vivant professores!
\item[$\boxtimes$] Vivat membrum quodlibet; Vivant membra quaelibet;
\item              $\lceil$ Semper sint in flore $\rfloor$.
\end{itemize}

\end{frame}

% XXXXXXXXXXXXXXXXXXXXXXXXXXXXXXXXXXXXXXXXXXXXXXXXXXXXXXXXXXXXXXXXXXXXXXXXXX
\end{document}

