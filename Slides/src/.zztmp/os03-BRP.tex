%%%%%%%%%%%%%%%%%%%%%%%%%%%%%%%%%%%%%%%%%%%%%%%%%%%%%%%%%%%%%%%%%%%%%%%%%
% REV154 Thu Aug 23 11:22:02 WIB 2018
% START0 Thu Jul 26 20:01:45 WIB 2018
%%%%%%%%%%%%%%%%%%%%%%%%%%%%%%%%%%%%%%%%%%%%%%%%%%%%%%%%%%%%%%%%%%%%%%%%%

\section{Week 03}
\begin{frame}[fragile]
\frametitle{Week 03 File System \& FUSE:
Topics\footnote{Source: ACM IEEE CS Curricula 2013}}

\begin{itemize}
\item Files: data, metadata, operations, organization, buffering, sequential, nonsequential
\item Directories: contents and structure
\item File systems: partitioning, mount/unmount, virtual file systems
\item Standard implementation techniques
\item Memory-mapped files
\item Special-purpose file systems
\item Naming, searching, access, backups
\item Journaling and log-structured file systems
\end{itemize}
\end{frame}

\begin{frame}[fragile]
\frametitle{Week 03 File System \& FUSE:
Learning Outcomes\footnote{Source: ACM IEEE CS Curricula 2013}}
\begin{itemize}
\item Describe the choices to be made in designing file systems. [Familiarity]
\item Compare and contrast different approaches to file organization, recognizing the strengths and weaknesses of each. [Usage]
\item Summarize how hardware developments have led to changes in the priorities for the design and the management of file systems. [Familiarity]
\item Summarize the use of journaling and how log-structured file systems enhance fault tolerance. [Familiarity]
\end{itemize}

\end{frame}

