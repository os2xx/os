%%%%%%%%%%%%%%%%%%%%%%%%%%%%%%%%%%%%%%%%%%%%%%%%%%%%%%%%%%%%%%%%%%%%%%%%
% Beamer Presentation - LaTeX - Template Version 1.0 (10/11/12)
% This template has been downloaded from: http://www.LaTeXTemplates.com
% License: % CC BY-NC-SA 3.0 (http://creativecommons.org/)
% Modified by Rahmat M. Samik-Ibrahim
% REV224 Mon Apr  6 00:35:14 WIB 2020
% REV156 Mon Aug 27 15:03:00 WIB 2018
% STARTX Mon Aug 13 20:51:13 WIB 2018
%%%%%%%%%%%%%%%%%%%%%%%%%%%%%%%%%%%%%%%%%%%%%%%%%%%%%%%%%%%%%%%%%%%%%%%%%

% PACKAGES AND THEMES ZCZC
\documentclass[xcolor=table, notheorems, hyperref={pdfpagelabels=false}]{beamer}
\input{beamer.tex}
\newcommand{\revision}{REV362 21-Nov-2021}
% w! tmptmp
% REV362 Sun 21 Nov 2021 17:54:07 WIB
% REV359 Sat 30 Oct 2021 14:42:29 WIB
% REV349 Sun 26 Sep 2021 09:13:27 WIB
% REV339 Sat 04 Sep 2021 12:50:36 WIB
% REV329 Tue 17 Aug 2021 20:15:00 WIB
% STARTS Wed 24 Aug 2016 19:34:33 WIB
%%%%%%%%%%%%%%%%%%%%%%%%%%%%%%%%%%%%%
\newcommand{\kopikopi}{\textcopyright{}2016-2021 VauLSMorg}



% XXXXXXXXXXXXXXXXXXXXXXXXXXXXXXXXXXXXXXXXXXXXXXXXXXXXXXXXXXXXXXXXXXXXXXXXXX
% The short title appears at the bottom of every slide, 
% the full title is only on the title page
% \date{\today}
\title[\textcopyright{}2016-2018 VauLSMorg]
{CSGE602055 Operating Systems \\ 
CSF2600505 Sistem Operasi \\
LeftOver}
\author{Rahmat M. Samik-Ibrahim}
\date{\revision}
\institute[UI]
{
University of Indonesia \\
\medskip
\texttt{http://os.vlsm.org/}
\\ \texttt{Always check for the latest revision!}
}

% XXXXXXXXXXXXXXXXXXXXXXXXXXXXXXXXXXXXXXXXXXXXXXXXXXXXXXXXXXXXXXXXXXXXXXXXXX
\begin{document}
\section{Start}
\begin{frame}
\titlepage
\end{frame}

% XXXXXXXXXXXXXXXXXXXXXXXXXXXXXXXXXXXXXXXXXXXXXXXXXXXXXXXXXXXXXXXXXXXXXXXXXX
\section{Agenda}
\begin{frame}{Outline}
  \frametitle{Agenda}
  \tableofcontents
\end{frame}

% XXXXXXXXXXXXXXXXXXXXXXXXXXXXXXXXXXXXXXXXXXXXXXXXXXXXXXXXXXXXXXXXXXXXXXXXXX
\section{LeftOver}
\begin{frame}
\frametitle{LeftOver}
\begin{itemize}
\item XXXX
\end{itemize}
\end{frame}

\begin{frame}[fragile]
\begin{itemize}
\item{The Check List (Operating Systems)}
\begin{itemize}
\item[$\square$] \textbf{Starting Point}: \texttt{http://os.vlsm.org/}
\item[$\square$] \textbf{Text Book}: any recent/decent OS book but map it to \textbf{OSC10}.
\item[$\square$] Create \textbf{public} project ''os182'' on your \texttt{github.com} account.
\begin{itemize}
\item[$\square$] Create file ''\texttt{README.md}'' and add an extra line every week. For e.g.\footnote{%
Week 00 line is optional. The following ''ZCZC WXX'' weekly tags are mandatory.}:
\begin{verbatim}
ZCZC Sistem Operasi 2018 Awal (1)
ZCZC W01 Have tried demo for week 01.
ZCZC W02 Week 02 is done.
ZCZC W03 Week 03 is done.
\end{verbatim}
\end{itemize}
\item[$\square$] Encode your \textbf{QRC} with image size of approximately 250x250 pixels:\\
                 {\scriptsize \textbf{''OS181 CLASS ID GITHUB-ACCOUNT SSO-ACCOUNT SIAK-Full-Name''}}\\
                 Special for Week 00: 
                 Mail your \textbf{embedded} QRC to: \texttt{operatingsystems@vlsm.org} with Subject: [W00] CLASS ID SIAK-NAME.
\item[$\square$] Write your Memo (with QRC) \textbf{every week}.
\item[$\square$] Using your \textbf{SSO} account, login to \texttt{badak.cs.ui.ac.id} via \texttt{kawung.cs.ui.ac.id}.
\begin{itemize}
\item[$\square$] Check folder \texttt{badak:///extra/Week00/}
\item[$\square$] Every week, copy the weekly demo files to your own home directory. Eg. for Week00:\\
                 \texttt{cp -r /extra/Week00/W00-demos/ W00-demos/}
\end{itemize}
\end{itemize}
\end{itemize}
\end{frame}

% XXXXXXXXXXXXXXXXXXXXXXXXXXXXXXXXXXXXXXXXXXXXXXXXXXXXXXXXXXXXXXXXXXXXXXXXXX
\section{Resources}
\begin{frame}
\frametitle{Resources}
\begin{itemize}
\item Text Book: any recent/decent OS book. Eg. (\textbf{OSC10}) Abraham Silberschatz, 
Peter B. Galvin, Greg Gagne: Operating System Concepts, $10^{th}$ Edition, 2018.
\item (GITHUB) \texttt{\textbf{https://github.com/UI-FASILKOM-OS/SistemOperasi/}}
\begin{itemize}
\item (DEMO)  --- \texttt{demos/}
\item (SLIDE) --- \texttt{pdf/}
\end{itemize}
\item (UJIAN) --- \texttt{\textbf{http://rms46.vlsm.org/2/195.pdf - 205.pdf}}
\item (BADAK) --- BADAK:\texttt{///extra/}
\end{itemize}
\end{frame}

% XXXXXXXXXXXXXXXXXXXXXXXXXXXXXXXXXXXXXXXXXXXXXXXXXXXXXXXXXXXXXXXXXXXXXXXXXX
\section{Bahan-bahan}
\begin{frame}
\frametitle{Bahan Presentasi:\\\texttt{http://rms46.vlsm.org/2/207.html}}
\begin{figure}
\includegraphics[width=0.55\linewidth]{os00-GIT}
\caption{\texttt{https://github.com/UI-FASILKOM-OS/os182/tree/master/pdf}}
\end{figure}
\end{frame}

% XXXXXXXXXXXXXXXXXXXXXXXXXXXXXXXXXXXXXXXXXXXXXXXXXXXXXXXXXXXXXXXXXXXXXXXXXX
\begin{frame}
\frametitle{Bahan Demo}
\begin{figure}
\includegraphics[width=0.73\linewidth]{os00-github-demo}
\caption{\texttt{https://github.com/UI-FASILKOM-OS/os182/tree/master/demos}}
\end{figure}
\end{frame}


% XXXXXXXXXXXXXXXXXXXXXXXXXXXXXXXXXXXXXXXXXXXXXXXXXXXXXXXXXXXXXXXXXXXXXXXXXX
\begin{frame}
\frametitle{Arsip SCELE}
\begin{figure}
\includegraphics[width=0.60\linewidth]{os00-arsip-scele}
\caption{Lihat juga BADAK.cs.ui.ac.id:///extra/}
\end{figure}
\end{frame}

% XXXXXXXXXXXXXXXXXXXXXXXXXXXXXXXXXXXXXXXXXXXXXXXXXXXXXXXXXXXXXXXXXXXXXXXXXX
\section{Accounts}
\begin{frame}
\frametitle{Github (New) Account 1}
\begin{figure}
\includegraphics[width=0.91\linewidth]{os00-akun-git-1}
\caption{Start a new project by ''rms46''.}
\end{figure}
\end{frame}

% XXXXXXXXXXXXXXXXXXXXXXXXXXXXXXXXXXXXXXXXXXXXXXXXXXXXXXXXXXXXXXXXXXXXXXXXXX
\begin{frame}
\frametitle{Github (New) Account 2}
\begin{figure}
\includegraphics[width=0.70\linewidth]{os00-akun-git-2}
\caption{Create public repository ''os182'' with a README.md file}
\end{figure}
\end{frame}

% XXXXXXXXXXXXXXXXXXXXXXXXXXXXXXXXXXXXXXXXXXXXXXXXXXXXXXXXXXXXXXXXXXXXXXXXXX
\begin{frame}
\frametitle{Github (New) Account 3}
\begin{figure}
\includegraphics[width=0.91\linewidth]{os00-akun-git-3}
\caption{Public project ''os182'' by ''rms46'' at https://github.com/rms46/os182}
\end{figure}
\end{frame}

% XXXXXXXXXXXXXXXXXXXXXXXXXXXXXXXXXXXXXXXXXXXXXXXXXXXXXXXXXXXXXXXXXXXXXXXXXX
\begin{frame}
\frametitle{WSL: Windows Subsystem for Linux}
\begin{figure}
\includegraphics[width=0.95\linewidth]{os00-wsl-ubuntu-00}
\caption{WSL: Windows Subsystem for Linux}
\end{figure}
\end{frame}

% XXXXXXXXXXXXXXXXXXXXXXXXXXXXXXXXXXXXXXXXXXXXXXXXXXXXXXXXXXXXXXXXXXXXXXXXXX
\begin{frame}
\frametitle{Cygwin}
\begin{figure}
\includegraphics[width=0.85\linewidth]{os00-cygwin-01}
\caption{Cygwin}
\end{figure}
\end{frame}

% XXXXXXXXXXXXXXXXXXXXXXXXXXXXXXXXXXXXXXXXXXXXXXXXXXXXXXXXXXXXXXXXXXXXXXXXXX
\section{Week 00: Problems}
\begin{frame}
\frametitle{Week 00: Problems}
\begin{itemize}
\item Tugas Minggu 00 (Week 00) ada dua:
\begin{itemize}
\item membuat QRC dan mengirimkannya via email.
\item membuat Memo Minggu 00 yang ada QRC, serta ditunjukkan pada saat istirahat kuliah hari ke dua.
\end{itemize}
\item ''TANDA PETIK'' BUKAN merupakan bagian dari QRC!
\item Jangan mencantumkan ''.git'' dan ''.sso'', jika bukan bagian dari nama akun anda!
\item Tanpa header [W00] pada Subject; email anda mungkin akan nyasar entah kemana...
      Ingat: [W00] (We-Nol-Nol) tidak sama dengan [WOO] (We-O-O)!
\item Ukuran QRC cukup sekitar 256 x 256 pixel: jangan terlalu besar atau terlalu kecil.
\item QRC ditanam (embedded) dalam email; jangan menggunakan attachment!
\item Jangan mengirim MEMO dalam format PDF!
\end{itemize}
\end{frame}

% XXXXXXXXXXXXXXXXXXXXXXXXXXXXXXXXXXXXXXXXXXXXXXXXXXXXXXXXXXXXXXXXXXXXXXXXXX
\section{Week 00: Check List}
\begin{frame}
\frametitle{Week 00: Check List}
\begin{itemize}
\item[$\square$] Starting \textbf{Week 01}: TABULA RASA is not accepted anymore!
\item[$\square$] Find/copy this document from \texttt{http://os.vlsm.org/}
\item[$\square$] Find/read a recent/decent OS Book and map it to \textbf{OSC10}.
\item[$\square$] Using your \textbf{SSO} account, login to \texttt{badak.cs.ui.ac.id} via \texttt{kawung.cs.ui.ac.id}.
\item[$\square$] Check folder \texttt{badak:///extra/Week00/}
\begin{itemize}
\item[$\square$] Try to copy and compile \texttt{c-program-example.c}.
\end{itemize}
\item[$\square$] QR Code: (Eg) ''\texttt{OS182 X 1253755125 demo Demo Suremo}''
\item[$\square$] Mailto: \texttt{operatingsystems@vlsm.org} \\
                 (Eg.) Subject:  \texttt{OS182 X 1253755125 demo Demo Suremo}
\item[$\square$] Write ''Memo Week00'' + your QRC.
\item[$\square$] \textbf{How to improve this document?}
\end{itemize}
\end{frame}

% 12 XXXXXXXXXXXXXXXXXXXXXXXXXXXXXXXXXXXXXXXXXXXXXXXXXXXXXXXXXXXXXXXXXXXXXXX
% XXXXXXXXXXXXXXXXXXXXXXXXXXXXXXXXXXXXXXXXXXXXXXXXXXXXXXXXXXXXXXXXXXXXXXXXXX
\section{The End}
\begin{frame}
\frametitle{The End}
\begin{itemize}
\item[$\square$] This is the end of the presentation.
\item[$\boxtimes$] This is the end of the presentation.
\item This is the end of the presentation.
\end{itemize}
\end{frame}






% XXXXXXXXXXXXXXXXXXXXXXXXXXXXXXXXXXXXXXXXXXXXXXXXXXXXXXXXXXXXXXXXXXXXXXXXXX
\section{Week 00: Self Service Assignments}
\begin{frame}[fragile]
\frametitle{Week 00: Self Service Assignments}
\begin{itemize}
\item Create project (\textbf{PUBLIC}) "os182" on your new (or existing) github.com account.
\item (Week 00) QRCode\footnote{%
''QR Code'' is a registered trademark and wordmark of Denso Wave Inc.}: 
''OS182\\CLASS ID GITHUB-ACCOUNT SSO-ACCOUNT SIAK-Full-Name''
\item (Weekly)  Memo.
\item Informasi Kuliah, Arsip Ujian, dan Demo
\begin{itemize}
\item \texttt{badak.cs.ui.ac.id:/extra/}
\item \texttt{https://github.com/UI-FASILKOM-OS/os182}
\item \texttt{https://rms46.vlsm.org/2/195.pdf} --- [195.pdf - 205.pdf].
\end{itemize}
\item Which BASH Account?
\begin{itemize}
\item Virtual Ubuntu: badak.cs.ui.ac.id (SSO)
\item Ubuntu (BYOD)
\item WSL: Windows 10 Subsystem for Linux
\item Cygwin (Windows)
\end{itemize}
\end{itemize}
\end{frame}


% XXXXXXXXXXXXXXXXXXXXXXXXXXXXXXXXXXXXXXXXXXXXXXXXXXXXXXXXXXXXXXXXXXXXXXXXXX
\section{Week 01: Problems}
\begin{frame}
\frametitle{Week 01: Problems}
\begin{itemize}
\item Tugas Minggu 01 (Week 01) ada dua:
\begin{itemize}
\item Memo Week01: to be checked  at break of the first lecture of WEEK 01.
\item Try Demo Week01 and write in ''README.md'' (os181), something like:
                 ''ZCZC W01 Telah mencoba demo Week01''.
\end{itemize}
\end{itemize}

\begin{figure}
\includegraphics[width=0.85\linewidth]{os01-README}
\caption{README.md: ZCZC W01 \dots}
\end{figure}

\end{frame}


% XXXXXXXXXXXXXXXXXXXXXXXXXXXXXXXXXXXXXXXXXXXXXXXXXXXXXXXXXXXXXXXXXXXXXXXXXX
\section{Week 02: Problems}
\begin{frame}
\frametitle{Week 02: Problems}
\begin{itemize}
\item Tugas Minggu 02 (Week 02) ada dua:
\begin{itemize}
\item Memo Week02: to be checked  at break of the first lecture of WEEK 01.
\item Try Demo Week02 and write in ''README.md'' (os181), something like:
                 ''ZCZC W02 Demo: done!''.
\end{itemize}
\end{itemize}

\begin{figure}
\includegraphics[width=0.80\linewidth]{os02-README}
\caption{README.md: ZCZC W02 Demo: done!}
\end{figure}

\end{frame}

% XXXXXXXXXXXXXXXXXXXXXXXXXXXXXXXXXXXXXXXXXXXXXXXXXXXXXXXXXXXXXXXXXXXXXXXXXX
\section{07-fork}
\begin{frame}[fragile]
\frametitle{07-fork}
% \begin{lstlisting}[basicstyle=\ttfamily\tiny]         % 108
% \begin{lstlisting}[basicstyle=\ttfamily\footnotesize] %  72
% \begin{lstlisting}[basicstyle=\ttfamily\small]        %  65
% \begin{lstlisting}[basicstyle=\ttfamily\large]        %  54
\begin{lstlisting}[basicstyle=\ttfamily\tiny]
>>>>> $ cat 07-fork.c 
/*
 * (c) 2005-2017 Rahmat M. Samik-Ibrahim
 * https://rahmatm.samik-ibrahim.vlsm.org/
 * This is free software.
 * REV05 Mon Oct 30 10:57:02 WIB 2017
 * REV02 Mon Oct 24 10:43:00 WIB 2016
 * REV01 Sun Feb 27 08:31:46 WIB 2011
 * START 2005
 */

#include <sys/types.h>
#include <sys/wait.h>
#include <stdio.h>
#include <stdlib.h>
#include <unistd.h>
#define DISPLAY1 "START * PARENT *** ** PID (%4d) ** **********\n"
#define DISPLAY2 "RANDOM: val1(%4d) -- val2(%4d) -- val3(%4d)\n"
/*************************************************** main ** */
void main(void) {
   pid_t val1, val2, val3;
   printf(DISPLAY1, getpid());
   val1 = fork();
   val2 = fork();
   val3 = fork();
   printf(DISPLAY2, val1, val2, val3);
   wait(NULL);
   wait(NULL);
   wait(NULL);
/* *********** START BLOCK ***
   *********** END * BLOCK *** */
}

\end{lstlisting}
\end{frame}

% XXXXXXXXXXXXXXXXXXXXXXXXXXXXXXXXXXXXXXXXXXXXXXXXXXXXXXXXXXXXXXXXXXXXXXXXXX
\begin{frame}[fragile]
\frametitle{07-fork (2)}
% \begin{lstlisting}[basicstyle=\ttfamily\tiny]         % 108
% \begin{lstlisting}[basicstyle=\ttfamily\footnotesize] %  72
% \begin{lstlisting}[basicstyle=\ttfamily\small]        %  65
% \begin{lstlisting}[basicstyle=\ttfamily\large]        %  54
\begin{lstlisting}[basicstyle=\ttfamily\tiny]

>>>>> $ 07-fork 
START * PARENT *** ** PID (6160) ** **********
RANDOM: val1(6161) -- val2(6162) -- val3(6163)
RANDOM: val1(6161) -- val2(6162) -- val3(   0)
RANDOM: val1(6161) -- val2(   0) -- val3(6165)
RANDOM: val1(6161) -- val2(   0) -- val3(   0)
RANDOM: val1(   0) -- val2(6164) -- val3(6166)
RANDOM: val1(   0) -- val2(6164) -- val3(   0)
RANDOM: val1(   0) -- val2(   0) -- val3(6167)
RANDOM: val1(   0) -- val2(   0) -- val3(   0)
>>>>> $ 07-fork 
START * PARENT *** ** PID (6168) ** **********
RANDOM: val1(6169) -- val2(6170) -- val3(6172)
RANDOM: val1(6169) -- val2(   0) -- val3(6173)
RANDOM: val1(6169) -- val2(6170) -- val3(   0)
RANDOM: val1(   0) -- val2(6171) -- val3(6174)
RANDOM: val1(6169) -- val2(   0) -- val3(   0)
RANDOM: val1(   0) -- val2(   0) -- val3(6175)
RANDOM: val1(   0) -- val2(   0) -- val3(   0)
RANDOM: val1(   0) -- val2(6171) -- val3(   0)
>>>>> $ 07-fork 
START * PARENT *** ** PID (6176) ** **********
RANDOM: val1(6177) -- val2(6178) -- val3(6181)
RANDOM: val1(   0) -- val2(6179) -- val3(6180)
RANDOM: val1(   0) -- val2(6179) -- val3(   0)
RANDOM: val1(   0) -- val2(   0) -- val3(6182)
RANDOM: val1(6177) -- val2(   0) -- val3(6183)
RANDOM: val1(6177) -- val2(   0) -- val3(   0)
RANDOM: val1(6177) -- val2(6178) -- val3(   0)
RANDOM: val1(   0) -- val2(   0) -- val3(   0)
>>>>> $ 
>>>>> $ 07-fork 

\end{lstlisting}
\end{frame}

% XXXXXXXXXXXXXXXXXXXXXXXXXXXXXXXXXXXXXXXXXXXXXXXXXXXXXXXXXXXXXXXXXXXXXXXXXX

% XXXXXXXXXXXXXXXXXXXXXXXXXXXXXXXXXXXXXXXXXXXXXXXXXXXXXXXXXXXXXXXXXXXXXXXXXX
\begin{frame}[fragile]
\frametitle{XXX}

\begin{itemize}
\item \textbf{4 SKS} (Units) means 12 hours per week!
\begin{itemize}
\item Ah Beng said: Work hard!
\end{itemize}
\item \textbf{No Lab --- No Task --- No Pop Quiz -- No Teaching Assistant}\footnotemark[\value{footnote}].
\begin{itemize}
\item No secret hand-shake!
\item But, it may vary from class to class.
\end{itemize}
\item \textbf{Active Preparation / Participation / Q\&A Only}.
\begin{itemize}
\item Pre-Midterm (UTS): 6 weeks @ 3 points (=18\%).
\item Post-Midterm: 5 weeks @ 3 points (=15\%).
\item Points for answering questions, trying demos, and writings memos.
\item Deductions for \textbf{NOT} answering questions: individually or collectively.
\end{itemize}
\end{itemize}

\end{frame}


% XXXXXXXXXXXXXXXXXXXXXXXXXXXXXXXXXXXXXXXXXXXXXXXXXXXXXXXXXXXXXXXXXXXXXXXXXX
\begin{frame}
\frametitle{Assessment part 2}

\begin{itemize}
\item \textbf{How to get points?}
\begin{itemize}
\item Answer questions, especially not in the middle of a lecture!
\item Try Demos.
\item Class attendance.
\end{itemize}
\item \textbf{MidTerm:} 6 set problems @ 6 points ( = 36\%).
\item \textbf{Final:} 30 points ( = 30\%)\footnote{THIS
       TERM ONLY. See
       \href{https://scele.cs.ui.ac.id/course/view.php?id=822}{SCELE}
       for more details!}.  \\
\item \textbf{Extra Rounding:} 1
%      point\footnote{Terms and Conditions apply. Void where prohibited by law.}  \\
       point\footnote{NOT AVAILABLE THIS TERM!}  \\
% \begin{itemize}
% \item Only if your grade is more than 59.0 and \textbf{ONE} more point.
% \end{itemize}
% \item \textbf{C-2C:} upto 5 points\footnotemark[\value{footnote}].
\item \textbf{C-2C:} upto 5 points \footnote{Terms
      and Conditions apply. Void where prohibited by law.}. \\
\begin{itemize}
\item Only if your grade is between 50.00 and 55.00 and you have a ''good'' track record.
\end{itemize}
\item Check your points regularly at \url{https://academic.ui.ac.id/} and
      \textbf{DO NOT COMPLAIN} weeks after! See also, \url{https://os.vlsm.org/}.
\end{itemize}

\end{frame}


% XXXXXXXXXXXXXXXXXXXXXXXXXXXXXXXXXXXXXXXXXXXXXXXXXXXXXXXXXXXXXXXXXXXXXXXXXXXXXXXXXXXXXXXXXXXXXXXXXXX
% 12 XXXXXXXXXXXXXXXXXXXXXXXXXXXXXXXXXXXXXXXXXXXXXXXXXXXXXXXXXXXXXXXXXXXXXXX
% XXXXXXXXXXXXXXXXXXXXXXXXXXXXXXXXXXXXXXXXXXXXXXXXXXXXXXXXXXXXXXXXXXXXXXXXXX
\section{The End}
\begin{frame}
\frametitle{The End}
\begin{itemize}
\item[$\square$] This is the end of the presentation.
\item[$\boxtimes$] This is the end of the presentation.
\item This is the end of the presentation.
\end{itemize}
\end{frame}

% XXXXXXXXXXXXXXXXXXXXXXXXXXXXXXXXXXXXXXXXXXXXXXXXXXXXXXXXXXXXXXXXXXXXXXXXXX
\end{document}

