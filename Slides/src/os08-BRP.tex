%%%%%%%%%%%%%%%%%%%%%%%%%%%%%%%
% REV412: Tue 22 Aug 2023 14:00
% REV406: Sat 05 Aug 2023 12:00
% REV154: Thu 23 Aug 2018 11:00
% START0: Thu 26 Jul 2018 20:00
%%%%%%%%%%%%%%%%%%%%%%%%%%%%%%%

\section{Week 08}
\begin{frame}[fragile]
\frametitle{Week 08 Scheduling:
Topics\footnote{Source: ACM IEEE CS Curricula}}

\begin{itemize}
\item Preemptive and non-preemptive scheduling 
\item Schedulers and policies
\item Processes and threads
\item Deadlines and real-time issues
\end{itemize}
\end{frame}

\begin{frame}[fragile]
\frametitle{Week 08 Scheduling:
Learning Outcomes\footnote{Source: ACM IEEE CS Curricula}}
\begin{itemize}
\item Compare and contrast the common algorithms used for both preemptive and non-preemptive scheduling of tasks in operating systems, such as priority, performance comparison, and fair-share schemes. [Usage]
\item Describe relationships between scheduling algorithms and application domains. [Familiarity]
\item Discuss the types of processor scheduling such as short-term, medium-term, long-term, and I/O.  [Familiarity]
\item Describe the difference between processes and threads. [Usage]
\item Compare and contrast static and dynamic approaches to real-time scheduling. [Usage]
\item Discuss the need for preemption and deadline scheduling. [Familiarity]
\item Identify ways that the logic embodied in scheduling algorithms are applicable to other domains, such as disk I/O, network scheduling, project scheduling, and problems beyond computing. [Usage]
\end{itemize}

\end{frame}

